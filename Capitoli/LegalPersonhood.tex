\section{The foundational roots of the legal personhood}

\subsection{Roman Law Origins}
In Roman Law, the concept of persona stemmed from theatrical masks, symbolizing an individual's legal status based on distinct characteristics. Status personae was conferred upon babies born to free fathers, emphasizing the human form, but excluding certain deformities or premature births. Interestingly, slaves were also recognized as "persons subjected to another's power," indicating a differentiated form of personhood.

\subsection{Middle Ages and Church Influence}
During the Middle Ages, the Church's influence contributed to the recognition of group personhood in Canon Law. This recognition allowed collective entities to possess legal separateness from individuals. Additionally, the notion of "personă moralis" emerged, suggesting that groups sharing common purposes could hold a distinct legal identity.

\subsection{Modern Development}
Over time, legal evolution delineated corporations and states as separate entities from their constituents, solidifying their status as legal persons. Simultaneously, the concept of "personă naturalis" developed, representing the individual human as a unified legal entity.

\subsection{Jurisprudential Perspectives}
From a jurisprudential standpoint, historical law connected the notion of legal "natural personhood" to human nature. Conversely, pure law jurisprudence defined "natural personhood" purely on legal identification without inherent ontological connections.

\section{Natural legal personhood and its recent, past 
developments}

\subsection{Codification in Statutory Law}
Based on the doctrinal thoughts which gradually constructed the concept of human “natural 
personality” as the general and abstract aptitude of an individual to be the holder of rights and 
bearer of duties from birth to death, the first time when this “natural personhood” was codified 
in a statutory law was the Article 16 of the Austrian Civil Code

\paragraph{Austrian Civil Code (ABGB)}
Article 16 of the ABGB promulgated in 1811 states: 
\begin{quote}
Jeder Mensch hat angeborne, schon durch die Vernunft einleuchtende Rechte, und ist daher als eine Person zu betrachten.
\end{quote}


\paragraph{Comparison of Different European Countries' Legal Codes}

\subparagraph{Recognition of Natural Personhood}
\begin{itemize}
    \item German Civil Code (BGB) recognizes legal capacity beginning at birth.
    \item Italian Civil Code confers legal personality to every person since birth.
    \item Portuguese Civil Code specifies that personality is acquired at the moment of complete birth with life.
    \item Greek Civil Code acknowledges the capacity for rights and obligations for every human being.
\end{itemize}

\subsection{International Law Perspectives}
\paragraph{UN Universal Declaration of Human Rights}

The UN Universal Declaration of Human Rights (UDHR) was adopted on December 10, 1948. It enshrines fundamental human rights and freedoms, aiming for universal recognition and protection.

\begin{itemize}[label=-]
    \item Article 1: "All human beings are born free and equal in dignity and rights."
    \item Article 6: "Everyone has the right to recognition everywhere as a person before the law."
\end{itemize}

\paragraph{African Charter on Human and Peoples' Rights}

The African Charter on Human and Peoples' Rights was adopted on January 19, 1981, by the Organization of African Unity (now African Union). It emphasizes the rights of individuals and communities across Africa.

\begin{itemize}[label=-]
    \item Article 5: "Every individual shall have the right to the respect of the dignity inherent in a human being and to the recognition of his legal status."
\end{itemize}

\paragraph{European Union Law Considerations}

European Union (EU) law incorporates principles of human rights, respecting the dignity and rights of individuals within the EU member states.

\begin{itemize}[label=-]
    \item The Charter of Fundamental Rights of the European Union (ECHR) emphasizes the foundational values of human dignity and rights.
    \item The Treaty on the European Union (TEU) outlines values such as respect for human dignity, freedom, democracy, and the rule of law.
\end{itemize}


\paragraph{Italian Legal System}
The Italian legal framework recognizes the concept of natural personhood, primarily through constitutional provisions.

\begin{itemize}[label=-]
    \item Article 2 of the Italian Constitution affirms the recognition and protection of the inviolable rights of man and requires the fulfillment of individual duties of solidarity in political, social, and economic realms.
    \item Article 22 prohibits the deprivation of legal capacity, citizenship, or name for political reasons, safeguarding individuals' legal status.
\end{itemize}
The Italian Civil Code explicitly acknowledges legal personality.

\begin{itemize}[label=-]
    \item Article 1 of the Civil Code grants legal personality to every individual from birth, establishing the capacity to hold rights and obligations.
\end{itemize}
The Italian Constitutional Court has issued rulings regarding embryos and assisted reproductive technologies.

\begin{itemize}[label=-]
    \item The Court distinguishes between the legal personhood of a born individual and the status of an embryo, considering the latter as a subject yet to acquire full personhood.
    \item Rulings like Case No. 27/1975 and No. 84/2016 emphasize the distinction between the rights of the mother as a legal person and the status of the embryo as a subject with a degree of subjectivity but not full personhood.
    \item Act No. 40/2004 addresses assisted reproductive technology, acknowledging the legal subjectivity of the conceived embryo during procedures but does not grant full personhood equivalent to a born individual.
\end{itemize}

\subsection{Non-Human Entities and Legal Personhood}
Some legal systems recognize legal personhood for non-human entities:

\begin{itemize}[label=-]
    \item Hindu law acknowledges legal entity status for religious idols based on religious customs.
    \item New Zealand's Te Urewera Act grants legal personhood to the Te Urewera river, providing it with rights, powers, duties, and liabilities.
    \item Austrian Civil Code recognizes animals as entities distinct from things and protects them through specific laws.
\end{itemize}

\paragraph{Italian Perspective on Animals and Environmental Protection}
The Italian legal system primarily considers animals as subjects of protection through statutory law.

\begin{itemize}[label=-]
    \item Constitutional Act No. 1/2022 amends Article 9 of the Italian Constitution to mandate safeguarding the environment, biodiversity, and ecosystems, including statutory regulations for animal protection.
    \item Article 9(3) emphasizes the obligation of the state legislator to regulate methods and forms of animal protection, indicating animals' status as passive recipients of protection rather than legal subjects.
\end{itemize}

\section{The robots (and the A.I. systems) as legal entities}
The granting of citizenship to the AI robot Sophia by Saudi Arabia was a symbolic but isolated event. International and Saudi Arabian Citizenship Laws typically reserve citizenship for "individuals" possessing "legal personality." This move hinted at potential recognition for artificial digital entities, marking a groundbreaking but singular instance. \newline
However, legal recognition of AI and robots varies across regions. Italian, EU laws, and other Western legal systems view them as "tools" or "objects," despite their advanced capabilities.
\newline
The European Parliament proposed "robotic legal subjectivity" for sophisticated autonomous robots responsible for their actions. Yet, the EU's Economic and Social Committee opposed this idea, citing risks of misuse and moral hazards.
\newline
UNESCO's Commission on Ethics criticized the concept of granting "electronic personhood" to AI, emphasizing the absence of human-like qualities such as free will, consciousness, moral agency, and intentionality, pivotal in legal personhood.

\subsection{Perspectives on Legal Personhood}
\subsubsection{Ontological Approach:}
Traditional legal personhood roots in natural human essence, posing challenges in ascribing it to AI or robots lacking human-like characteristics.

\subsubsection{Refuting the Ontological Approach:}
Juristic persons lack natural human traits yet hold legal personhood, driven by internal mechanisms for decision-making. Recognition of legal personhood for incapacitated individuals challenges strict adherence to human-like characteristics.

\subsubsection{Legalistic Approach:}
Legal personhood becomes a construct conferred based on legal identification as rights-holders, allowing flexibility for any entity to potentially gain legal personhood.

\subsection*{Criteria for Recognizing Legal Personhood in Robots or AI}

\subsubsection{Criteria:}
\begin{itemize}[label=-]
    \item \textbf{Technological sophistication:} High complexity and capacity for internal decision-making.
    \item \textbf{Autonomy and adaptability:} Ability to process information autonomously, beyond pre-programmed routines.
    \item \textbf{Deep learning mechanisms:} Capacity to learn and adapt.
\end{itemize}

\subsubsection{Challenges and Considerations:}
\begin{itemize}[label=-]
    \item Allocation of resources, punishment for unlawful behavior.
    \item Recognition based on societal and economic benefits rather than strict criteria.
    \item Accountability and the chain of responsibility between involved parties.
\end{itemize}

\subsection{Human Dignity Consideration}

Human dignity, entrenched in legal systems, is intrinsically linked to human life and nature. Robots, despite advancements, lack the essence of life that dignifies humanity. Envisioning a future where robots surpass humanity poses significant threats, necessitating a distinct legal framework to avert risks from their autonomous intelligence.
\newline
Dignity is an exclusive attribute of humanity, incapable of being shared with artificial entities. Establishing a differentiated legal regime for intelligent artifacts becomes imperative to protect against potential harm from these machines.
\newline
Robots, lacking dignity, cannot be accorded the same "natural personhood" or "juristic personhood" as humans or corporations, as the latter are controlled by humans, unlike autonomous robots making decisions.
\newline
In conclusion, robots, devoid of dignity, require a differentiated legal status. Granting them the same status as humans undermines the fundamental distinctions between artificial and human existence.

\section{Redesigning Legal Personhood: Biomedical and Digital Transitions}
The concept of legal personality is evolving to meet scientific and technological progress. Two significant transitions are evident:

\subsection{Biomedical Transition}
The legal system bifurcates natural personhood into "natural subjectivity" for prenatal life and "natural personhood" post-birth. Both segments uphold human dignity and fundamental rights, yet they differ in their legal equivalence.

\subsection{Digital Transition}
The discussion emerges regarding a new legal position for robots, distinct from their current classification as mere objects. While existing laws may view robots as manufactured products, the possibility of a separate legal status is contemplated.
\newline
However, granting robots a distinct legal status raises complexities. Robots, as inanimate artifacts, cannot derive legal subjectivity from human attributes, nor can they be equated with juristic persons (corporations). Crafting a new legal agentivity status for advanced robots is proposed as a speculative idea.

\subsection{Hypothetical Legal Statuses}
\subsubsection{Legal Personhood}
Remains the universal status for living humans, denoting full rights, duties, and dignity.
\subsubsection{Legal Subjectivity}
Attributed to entities not considered legal persons but capable of self-development into newborns, such as embryos. This status may recognize their dignity and specific rights.
\subsubsection{Legal Agentivity}
Hypothetical status for advanced robots, distinguishing them from humans and juristic subjects due to their autonomous decision-making and learning abilities.
\newline
This proposed status aims to acknowledge robots' sophistication without equating them to legal objects, recognizing their capabilities and granting limited rights and responsibilities.
\newline
\newline
In conclusion, despite the possibility of introducing a new legal position for robots, the legal landscape post-digital and biomedical transitions remains complex and uncharted. The law must continuously adapt to these panoramas while preserving the fundamental principle of "human-centrism," safeguarding human dignity, rights, societal well-being, and democracy.

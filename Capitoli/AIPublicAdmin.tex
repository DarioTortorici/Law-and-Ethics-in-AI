\section{Introduction: Digitalization and AI in the Public Sector}
Already in 1979, a famous report on the main problems of the Italian State administration (the so-called Giannini report) recommended implementing the use of electronic calculators for a general modernization of the bureaucratic machinery. Similar developments occurred throughout Europe, with the digitalization reform process being encouraged by European institutions. The strategic relevance of digitalizing the public sector is strongly emphasized by the Italian National Plan of Recovery and Resiliency (PNRR), aligned with other National Plans pursuing the digital transition objective set by EU Regulation 2021/241 (which instituted the Recovery and Resiliency Facility).

Moving from the normative dimension to administrative practice, the digitalization of the public sector has not only increased quantitatively but also qualitatively. This ongoing change marks a paradigm shift from "formal digitalization" towards "substantial digitalization," increasingly involving the use of AI.

\section{The Use of AI by Public Administrations}

Regarding various input data, AI can provide:

\begin{enumerate}[label=\alph*)]
    \item a description, telling what happened or the features of a given dataset (e.g., clustering, classifying, etc.);
    \item a diagnosis, explaining why something happened (causal inferences);
    \item a prediction, forecasting statistically probable future events (statistical inferences);
    \item a prescription, capable of making actual decisions and implementations.
\end{enumerate}

To illustrate the breadth of the phenomenon, consider some examples:

\subsection{Monitoring Compliance and Regulatory Enforcement}

Public authorities are tasked with ensuring compliance with applicable laws by citizens and enterprises. For instance, a local authority monitors enterprises authorized to conduct specific commercial activities, enforcing pre-established technical and safety rules. Similarly, financial market authorities verify compliance with financial reporting rules. Machine learning algorithms can assess risks and predict potential misconduct based on past enforcement actions or significant changes in earnings.

Different algorithms target first-time violators versus serial violators, helping authorities save time. However, relying solely on algorithmic results might lead to overlooking contextual nuances.

\subsection{Public Security and Law Enforcement Functions}

Custom authorities, airport security, and police forces globally have utilized biometric recognition systems, particularly facial recognition through deep learning in cameras. These systems can:

\begin{enumerate}[label=\roman*)]
    \item identify individuals by matching images with existing profiles (one-to-many matching);
    \item verify the identity of specific individuals (one-to-one matching);
    \item predict live risks related to the intentions and moods of supervised individuals, aiding in law enforcement decisions.
\end{enumerate}

Facial recognition facilitates quicker responses in emergencies like terrorism prevention but poses risks such as privacy concerns, biases, and discrimination in predictive applications.

\section{Risks and Legal Issues}

A preliminary question addressed by several observers is whether public authorities have a duty to base the use of AI on specific legal foundations. Moving on to specific legal issues related to AI in the public sector:

\subsection{Transparency and Explainability of Outcomes}

An initial concern involves transparency and the explainability of outcomes derived from AI systems.

\subsection{Technical Malfunctions of AI Platforms}

Another issue pertains to technical malfunctions within AI-integrated platforms. While these malfunctions aren't entirely novel, they remain a concern within AI-specific contexts.

\subsection{Use of AI in Discretionary Powers}

A more fundamental issue arises when public authorities are vested with discretionary powers. Recent changes in Italian jurisprudence, notably in the interpretation of constitutional Article 97, have shifted from impeding technological change to fostering innovation and AI system implementation in the public sector.

Several other issues require consideration in the experimentation and use of AI in the public sector. For instance, the development of AI systems for public authorities, the efficacy of anonymization and "blinding" strategies, among others.

\section{Regulation of AI in the Public Sector}

Seeking rules for "algorithmic legality," there's a quest for a new paradigm based on principles outlined in the EU General Data Protection Regulation (GDPR). Article 22 establishes the right not to be solely subject to automated decisions with legal or similarly significant effects.

However, this rule presents significant exceptions. Paragraph 2 allows automated decisions if:

\begin{enumerate}
    \item necessary for entering into or performing a contract;
    \item provided by EU or State law with measures to safeguard data subject rights;
    \item explicit consent is given by the data subject.
\end{enumerate}

Especially noteworthy is the third exception, potentially leading individuals to unknowingly provide consent along with general contractual conditions.

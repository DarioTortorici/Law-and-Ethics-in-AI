\section{What is Health? Definition and Protection from the Legal Perspective}
Health is defined by the World Health Organization (WHO) as "a state of complete physical, mental, and social well-being and not merely the absence of disease or infirmity." The importance of health, along with the right to health, is acknowledged not only at the international level but also within the European and national contexts. For instance, Article 35 of the Charter of Fundamental Rights of the European Union declares that "everyone has the right of access to preventive health care and the right to benefit from medical treatment under the conditions established by national laws and practices. A high level of human health protection shall be ensured in the definition and implementation of all Union policies and activities."

\section{The Rise of Artificial Intelligence in the Medical Field}
The medical field stands as one of the most impacted sectors with the integration of AI. Within healthcare, AI systems serve as the primary tools facilitating a paradigm shift. This shift outlines a new medical model emphasizing self-managed health, well-being, prevention, and precision medicine rather than focusing solely on disease and standardized treatment protocols.

\subsection{Looking for an Artificial Doctor}
Three primary benefits can be identified with the application of AI systems. Firstly, this technology significantly enhances the efficiency of doctor-patient relationships from temporal, efficacy, and economic perspectives. It offers a different dimension in the availability of time for care and elevates efficacy through intelligent systems, increasing diagnostic accuracy, reducing human errors, avoiding unnecessary clinical procedures, and providing updated medical information. Economically, AI allows for delegated healthcare functions, potentially reducing healthcare service costs.
\newline
Secondly, AI enables the personalization of diagnoses and therapeutic treatments. Tailoring therapies based on individual reactions to diseases fosters a more precise, accurate, and responsive medical model, benefiting patient health.
\newline
Thirdly, AI application empowers patients in their medical decisions. Access to clinical information, monitoring and enhancing personal medical conditions, understanding physical and psychological needs, and tailoring treatments to daily routines all contribute to placing the patient at the center of the therapeutic dimension.
\newline
Furthermore, the ability to engage with a virtual medical expert, comprehend the impact of everyday actions on clinical conditions, and reduce hospitalization by promoting treatments in familiar environments all contribute to enhancing patients' awareness of their health.

\subsection{Towards a New Technological Paternalism}
Despite the significant benefits arising from the extensive use of intelligent technologies in the medical field, several problematic aspects emerge from the integration of AI into the therapeutic dimension. Four critical profiles influenced by AI in the decision-making process within the doctor-patient relationship need consideration.
\newline
Firstly, concerns arise regarding data quality and usage, which become pivotal in addressing the legal implications of AI application. The risk of feeding poorly varied, inaccurate, or unrepresentative data to AI, as well as data reproducing human errors and biases, can compromise the reliability and fairness of results used in patients' diagnosis, treatment, and monitoring. Issues also revolve around accessibility, sharing, and ownership of healthcare data crucial for the proper functioning of these technologies and the protection of patients' privacy.
\newline
Secondly, the deployment of AI may exacerbate new forms of technological divide. Lack of proficiency among medical professionals in utilizing AI, from both economic and digital education perspectives, could exclude patients from reaping the benefits of this technology.
\newline
Thirdly, the opacity of AI poses risks. Despite its ability to produce accurate results, AI's lack of explainability may hinder understanding the logical steps behind specific decisions. This lack of transparency might challenge healthcare professionals in validating or explaining clinical hypotheses and therapeutic proposals, potentially eroding patient trust in their doctors' interpretation and communication of AI-generated choices.
\newline
Finally, the proliferation of AI in the therapeutic dimension might lead to the deskilling of healthcare professionals. Over-reliance on intelligent technologies could diminish doctors' skills and expertise over time, potentially blurring their awareness of essential human aspects and factors vital for providing effective and personalized care.

\section{Balancing benefits and risks}
The proposal of the EU Regulation on Artificial Intelligence, also known 
as \hyperref[sec:AIAct]{AI Act}, can offer an interesting regulatory solution to face the problematic issues arising 
from the application of AI within the medical dimension, and this although the limited 
competences and the contents of the proposal of the \hyperref[sec:AIAct]{AI Act} may not offer an effective and full 
protection to fundamental rights safeguarded within the legal framework of the therapeutic 
relationship
